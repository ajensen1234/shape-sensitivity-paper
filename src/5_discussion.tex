\section{Discussion}
The findings correspond closely with initial expectations regarding the sensitivity measurement of projected shapes relative to 3D object orientation and are consistent with areas challenging for JTML optimization.
Specifically, the humeral implant showed a generally smooth and minimal shape sensitivity profile, particularly for $\delta_{y}$ rotations (\cref{tab:ss-vals}).
Along this axis, the humeral implant is the most cylindrical, meaning we would not expect to see a significant change in the shape descriptor with minor $\delta_{y}$ rotations.
Furthermore, it is noteworthy that this axis presented the most significant difficulties in JTML optimization.

Similar intuitive outcomes are observed with the glenosphere implant, which exhibited the lowest average $\mathbb{S}(\delta)$ among all the implant types.
This implant primarily consists of an articulation surface closely approximating a spherical shape.
Given that the projection of a sphere (a circle) remains constant regardless of the sphere's orientation, the closer a shape is to a spherical form, the lower its overall shape sensitivity is expected to be.

The observed shape sensitivity of the tibial implant with respect to $\delta_{y}$ rotation aligns with the concept of symmetry traps.
There is a consistently low shape sensitivity along the line where $x=0$.
This axis, associated with internal/external rotation, is the same one that contributes to symmetry traps, where two different 3D orientations result in an identical projected shape.
In terms of this analysis, the $\Delta S$ value would be $0$ for these two orientations of the tibial implant.

This study sheds light on an important aspect of Joint Track Machine Learning, particularly the use of Euler angles in the DIRECT-JTA optimization routine.
Currently, the optimization does not involve independently varying all angles within a body-centered reference frame, as this approach is not conducive to hyperbox creation.
Instead, optimization is performed over a range of ordered rotations, projected through the sequence $R_{z}R_{x}R_{y}$.
The challenges the humeral implant encounters in aligning the $y$-axis illustrate that this ordered sequence, especially with a symmetric final axis, can hinder the convergence process.

Beyond the inherent shape sensitivities, such optimization limitations motivate exploring alternatives to Euler angles.
Performing registration optimization directly on the Special Orthogonal group $SO(3)$ poses an intriguing direction.
$SO(3)$ encapsulates all possible 3D rotations in a mathematically convenient structure (A \emph{Lie Group}, which is both a manifold and a group).
By optimizing on this manifold instead of using specific angle parametrizations, issues with gimbal lock and cascade effects can be avoided.
Optimization over Lie groups is an emerging subfield - establishing robust $SO(3)$-based registration cost functions could significantly improve JTML convergence while relying less on descriptor sensitivity along certain axes.

Lastly, incorporating bone information into the optimization routine seems to be the most likely path forward for disambiguating difficult implant orientations.
This study demonstrates that for specific implant geometries, there are inherent limitations to the information that can be gathered from the projected 2D shape as manifest in fluoroscopic imaging.
Utilizing bone information, such as keypoints representing specific anatomic structures, would drastically increase the amount of information present during optimization, and could offer robust constraints on the possible 3D orientations of implant models.



%%% Local Variables:
%%% mode: latex
%%% TeX-master: "../Jensen_Shape_Sensitivity"
%%% End:
